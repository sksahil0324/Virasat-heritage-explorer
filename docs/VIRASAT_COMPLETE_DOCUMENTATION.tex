\documentclass[12pt,a4paper]{report}
\usepackage[utf-8]{inputenc}
\usepackage[margin=1in]{geometry}
\usepackage{graphicx}
\usepackage{tikz}
\usepackage{listings}
\usepackage{xcolor}
\usepackage{hyperref}
\usepackage{booktabs}
\usepackage{float}
\usepackage{fancyhdr}
\usepackage{lastpage}

% Code styling
\lstset{
    language=JavaScript,
    basicstyle=\ttfamily\small,
    keywordstyle=\color{blue},
    commentstyle=\color{gray},
    stringstyle=\color{red},
    breaklines=true,
    showstringspaces=false,
    tabsize=2,
    frame=single,
    backgroundcolor=\color{gray!10}
}

% Header and Footer
\pagestyle{fancy}
\fancyhf{}
\rhead{VIRASAT - Heritage Explorer}
\lhead{\leftmark}
\cfoot{Page \thepage\ of \pageref{LastPage}}

\title{\textbf{VIRASAT - Heritage Explorer}\\
\large{Complete Technical Documentation}\\
\normalsize{Digital Platform for Indian Heritage Sites}}
\author{Development Team}
\date{\today}

\begin{document}

\maketitle

\tableofcontents
\newpage

% ============================================================================
% CHAPTER 1: INTRODUCTION
% ============================================================================
\chapter{Introduction}

\section{Project Overview}

VIRASAT (meaning ``Heritage'' in Hindi) is an innovative digital platform designed to bridge the gap between India's rich cultural heritage and modern technology. The platform serves as a comprehensive virtual museum and exploration tool, enabling users worldwide to discover, learn about, and experience India's magnificent heritage sites through immersive digital experiences.

\subsection{Vision and Mission}

The primary vision of VIRASAT is to democratize access to India's cultural heritage by leveraging cutting-edge web technologies. The mission encompasses:

\begin{itemize}
    \item \textbf{Preservation}: Digital documentation and preservation of heritage sites for future generations
    \item \textbf{Education}: Providing comprehensive historical and cultural information to educate users
    \item \textbf{Accessibility}: Making heritage sites accessible to everyone, regardless of physical location or mobility constraints
    \item \textbf{Innovation}: Utilizing modern web technologies to create engaging, immersive experiences
    \item \textbf{Community}: Building a community of heritage enthusiasts and researchers
\end{itemize}

\section{Problem Statement}

India possesses one of the world's richest cultural heritages, with thousands of monuments, temples, forts, and archaeological sites. However, several challenges limit public engagement:

\begin{enumerate}
    \item \textbf{Geographic Barriers}: Many heritage sites are located in remote areas, making physical visits difficult
    \item \textbf{Limited Information}: On-site information is often minimal or not available in multiple languages
    \item \textbf{Preservation Concerns}: Physical tourism can contribute to wear and degradation of ancient structures
    \item \textbf{Accessibility Issues}: People with mobility constraints face difficulties visiting sites
    \item \textbf{Fragmented Resources}: Information about heritage sites is scattered across multiple sources
\end{enumerate}

VIRASAT addresses these challenges by providing a centralized, accessible, and immersive digital platform.

\section{Key Features}

\begin{table}[H]
\centering
\caption{VIRASAT Platform Features Overview}
\begin{tabular}{@{}lp{8cm}@{}}
\toprule
\textbf{Feature Category} & \textbf{Description} \\
\midrule
Exploration & Browse sites via list/grid view, interactive map, search, and filters \\
Immersive Views & 360° panoramas, 3D models, high-quality media galleries \\
Audio Guides & Multi-language audio summaries with play tracking \\
User Features & Authentication, favorites, personalized dashboard \\
Interactive Map & Leaflet-based map with GeoJSON, custom markers, state selection \\
Admin Dashboard & Content management, media uploads, analytics, user management \\
Responsive Design & Optimized for desktop, tablet, and mobile devices \\
Futuristic Theme & Glass morphism, holographic effects, particle animations \\
\bottomrule
\end{tabular}
\end{table}

% ============================================================================
% CHAPTER 2: TECHNOLOGY STACK
% ============================================================================
\chapter{Technology Stack and Architecture}

\section{Frontend Technologies}

\begin{table}[H]
\centering
\caption{Frontend Technology Stack}
\begin{tabular}{@{}llp{6cm}@{}}
\toprule
\textbf{Technology} & \textbf{Version} & \textbf{Purpose} \\
\midrule
React & 19.1.0 & UI component library and rendering \\
TypeScript & 5.8.3 & Type-safe JavaScript development \\
Vite & 6.3.5 & Build tool and development server \\
React Router & 7.6.1 & Client-side routing \\
Tailwind CSS & 4.1.8 & Utility-first CSS framework \\
Framer Motion & 12.15.0 & Animation and transitions \\
Shadcn UI & Latest & Pre-built UI components \\
Leaflet & 1.9.4 & Interactive mapping \\
Three.js & Latest & 3D rendering \\
\bottomrule
\end{tabular}
\end{table}

\section{Backend Technologies}

\begin{table}[H]
\centering
\caption{Backend Technology Stack}
\begin{tabular}{@{}llp{6cm}@{}}
\toprule
\textbf{Technology} & \textbf{Version} & \textbf{Purpose} \\
\midrule
Convex & 1.27.0 & Backend-as-a-Service, database, and API \\
Convex Auth & Latest & Authentication with OTP \\
Node.js & 18+ & Runtime for actions \\
\bottomrule
\end{tabular}
\end{table}

\section{System Architecture}

\subsection{High-Level Architecture}

The VIRASAT platform follows a modern three-tier architecture:

\begin{enumerate}
    \item \textbf{Presentation Layer}: React-based frontend with responsive UI components
    \item \textbf{Application Layer}: Convex backend with queries, mutations, and actions
    \item \textbf{Data Layer}: Convex NoSQL database with optimized indexes
\end{enumerate}

\subsection{Component Structure}

\begin{itemize}
    \item \textbf{Pages}: Landing, Explore, SiteDetail, AdminDashboard, Auth, Favorites, NotFound
    \item \textbf{Components}: ParticleBackground, HolographicCard, InteractiveMap, Model3DViewer, PanoramaViewer
    \item \textbf{UI Components}: 50+ Shadcn UI components for consistent design
    \item \textbf{Hooks}: useAuth, useMobile, and custom hooks for shared logic
    \item \textbf{Utilities}: Helper functions for downloads, formatting, and data manipulation
\end{itemize}

% ============================================================================
% CHAPTER 3: DATABASE SCHEMA
% ============================================================================
\chapter{Database Schema and Data Model}

\section{Schema Overview}

The VIRASAT database uses Convex's NoSQL document model with strong typing through validators. The schema defines five main tables with strategic indexing for optimal query performance.

\section{Table Definitions}

\subsection{Users Table}

\begin{lstlisting}[caption=Users Table Schema]
users: defineTable({
  name: v.optional(v.string()),
  image: v.optional(v.string()),
  email: v.optional(v.string()),
  emailVerificationTime: v.optional(v.number()),
  isAnonymous: v.optional(v.boolean()),
  role: v.optional(roleValidator), // "admin" | "user" | "member"
}).index("email", ["email"])
\end{lstlisting}

\textbf{Purpose}: Stores user account information and authentication data.

\textbf{Key Fields}:
\begin{itemize}
    \item \texttt{role}: Determines access level (admin has full control, user has limited access)
    \item \texttt{email}: Unique identifier for authentication
    \item \texttt{isAnonymous}: Tracks anonymous user sessions
\end{itemize}

\subsection{Heritage Sites Table}

\begin{lstlisting}[caption=Heritage Sites Table Schema]
heritageSites: defineTable({
  name: v.string(),
  description: v.string(),
  historicalSignificance: v.string(),
  category: categoryValidator,
  state: v.string(),
  city: v.string(),
  latitude: v.optional(v.number()),
  longitude: v.optional(v.number()),
  isUNESCO: v.boolean(),
  timePeriod: v.optional(v.string()),
  visitorGuidelines: v.optional(v.string()),
  viewCount: v.number(),
  isPublished: v.boolean(),
  createdBy: v.id("users"),
  folkTales: v.optional(v.string()),
  culturalHeritage: v.optional(v.string()),
  cuisine: v.optional(v.string()),
  stories: v.optional(v.string()),
  community: v.optional(v.string()),
  ticketPrice: v.optional(v.string()),
  openingHours: v.optional(v.string()),
  bestTimeToVisit: v.optional(v.string()),
  timezone: v.optional(v.string()),
  view360Url: v.optional(v.string()),
  view3dUrl: v.optional(v.string()),
})
.index("by_state", ["state"])
.index("by_category", ["category"])
.index("by_published", ["isPublished"])
.index("by_unesco", ["isUNESCO"])
.index("by_view_count", ["viewCount"])
\end{lstlisting}

\textbf{Purpose}: Core table storing all heritage site information.

\textbf{Categories}: temple, fort, palace, monument, museum, archaeological, natural, other

\textbf{Indexing Strategy}:
\begin{itemize}
    \item \texttt{by\_published}: Fast retrieval of public sites
    \item \texttt{by\_state}: Geographic filtering
    \item \texttt{by\_category}: Category-based browsing
    \item \texttt{by\_unesco}: UNESCO site filtering
    \item \texttt{by\_view\_count}: Popularity sorting
\end{itemize}

\subsection{Media Table}

\begin{lstlisting}[caption=Media Table Schema]
media: defineTable({
  siteId: v.id("heritageSites"),
  type: v.union(
    v.literal("image"),
    v.literal("video"),
    v.literal("model3d"),
    v.literal("panorama")
  ),
  storageId: v.optional(v.id("_storage")),
  url: v.string(),
  caption: v.optional(v.string()),
  isPrimary: v.boolean(),
}).index("by_site", ["siteId"])
\end{lstlisting}

\textbf{Purpose}: Stores media files associated with heritage sites.

\textbf{Media Types}:
\begin{itemize}
    \item \texttt{image}: High-resolution photographs
    \item \texttt{video}: Documentary or promotional videos
    \item \texttt{model3d}: Interactive 3D models
    \item \texttt{panorama}: 360° panoramic views
\end{itemize}

\textbf{Key Features}:
\begin{itemize}
    \item \texttt{isPrimary}: Designates the primary image for site display
    \item \texttt{storageId}: References Convex file storage
    \item \texttt{url}: Direct access URL for the media file
\end{itemize}

\subsection{Audio Summaries Table}

\begin{lstlisting}[caption=Audio Summaries Table Schema]
audioSummaries: defineTable({
  siteId: v.id("heritageSites"),
  storageId: v.id("_storage"),
  url: v.string(),
  duration: v.optional(v.number()),
  language: v.string(),
  playCount: v.number(),
}).index("by_site", ["siteId"])
\end{lstlisting}

\textbf{Purpose}: Stores audio guide information for heritage sites.

\textbf{Features}:
\begin{itemize}
    \item Multi-language support
    \item Play count tracking for analytics
    \item Duration metadata for UI display
\end{itemize}

\subsection{Favorites Table}

\begin{lstlisting}[caption=Favorites Table Schema]
favorites: defineTable({
  userId: v.id("users"),
  siteId: v.id("heritageSites"),
})
.index("by_user", ["userId"])
.index("by_site", ["siteId"])
.index("by_user_and_site", ["userId", "siteId"])
\end{lstlisting}

\textbf{Purpose}: Tracks user favorite sites for personalized experience.

\textbf{Indexes}:
\begin{itemize}
    \item \texttt{by\_user}: Retrieve all favorites for a user
    \item \texttt{by\_site}: Find users who favorited a site
    \item \texttt{by\_user\_and\_site}: Check if a specific site is favorited
\end{itemize}

% ============================================================================
% CHAPTER 4: BACKEND FUNCTIONS
% ============================================================================
\chapter{Backend Functions and API}

\section{Heritage Sites Functions}

\subsection{Query: list}

\begin{lstlisting}[caption=List Published Heritage Sites]
export const list = query({
  args: {
    category: v.optional(v.string()),
    state: v.optional(v.string()),
    unescoOnly: v.optional(v.boolean()),
  },
  handler: async (ctx, args) => {
    // Fetch published sites with optional filters
    // Returns sites sorted by view count
  },
});
\end{lstlisting}

\textbf{Purpose}: Retrieve published heritage sites with optional filtering.

\textbf{Parameters}:
\begin{itemize}
    \item \texttt{category}: Filter by site category
    \item \texttt{state}: Filter by Indian state
    \item \texttt{unescoOnly}: Show only UNESCO World Heritage Sites
\end{itemize}

\textbf{Returns}: Array of sites with associated media, sorted by popularity.

\subsection{Query: getById}

\begin{lstlisting}[caption=Get Single Heritage Site]
export const getById = query({
  args: { id: v.id("heritageSites") },
  handler: async (ctx, args) => {
    // Fetch site with media and audio
  },
});
\end{lstlisting}

\textbf{Purpose}: Retrieve detailed information for a single heritage site.

\textbf{Returns}: Site object with all associated media and audio guides.

\subsection{Mutation: create}

\begin{lstlisting}[caption=Create Heritage Site (Admin Only)]
export const create = mutation({
  args: {
    name: v.string(),
    description: v.string(),
    historicalSignificance: v.string(),
    category: v.union(...),
    state: v.string(),
    city: v.string(),
    // ... additional fields
  },
  handler: async (ctx, args) => {
    // Verify admin role
    // Insert new site
  },
});
\end{lstlisting}

\textbf{Purpose}: Create a new heritage site (admin only).

\textbf{Authorization}: Requires admin role.

\subsection{Mutation: update}

\begin{lstlisting}[caption=Update Heritage Site (Admin Only)]
export const update = mutation({
  args: {
    id: v.id("heritageSites"),
    // ... optional fields to update
  },
  handler: async (ctx, args) => {
    // Verify admin role
    // Update site fields
  },
});
\end{lstlisting}

\textbf{Purpose}: Update existing heritage site information.

\textbf{Authorization}: Requires admin role.

\subsection{Mutation: remove}

\begin{lstlisting}[caption=Delete Heritage Site (Admin Only)]
export const remove = mutation({
  args: { id: v.id("heritageSites") },
  handler: async (ctx, args) => {
    // Delete associated media and audio
    // Delete site
  },
});
\end{lstlisting}

\textbf{Purpose}: Delete a heritage site and all associated content.

\textbf{Authorization}: Requires admin role.

\textbf{Cascade Deletion}: Automatically deletes all media and audio files.

\section{Media Management Functions}

\subsection{Mutation: addMedia}

\begin{lstlisting}[caption=Add Media to Site]
export const addMedia = mutation({
  args: {
    siteId: v.id("heritageSites"),
    type: v.union(...),
    storageId: v.optional(v.id("_storage")),
    url: v.string(),
    caption: v.optional(v.string()),
  },
  handler: async (ctx, args) => {
    // Verify admin role
    // Insert media record
  },
});
\end{lstlisting}

\textbf{Purpose}: Add media file to a heritage site.

\subsection{Mutation: removeMedia}

\begin{lstlisting}[caption=Remove Media from Site]
export const removeMedia = mutation({
  args: { mediaId: v.id("media") },
  handler: async (ctx, args) => {
    // Verify admin role
    // Delete media record
  },
});
\end{lstlisting}

\textbf{Purpose}: Remove media file from a site.

\subsection{Mutation: setPrimaryMedia}

\begin{lstlisting}[caption=Set Primary Media]
export const setPrimaryMedia = mutation({
  args: {
    siteId: v.id("heritageSites"),
    mediaId: v.id("media"),
  },
  handler: async (ctx, args) => {
    // Verify admin role
    // Update isPrimary flags
  },
});
\end{lstlisting}

\textbf{Purpose}: Set a media file as the primary image for a site.

\textbf{Logic}: Ensures only one primary media per site.

\section{User and Authentication Functions}

\subsection{Query: getCurrentUser}

\begin{lstlisting}[caption=Get Current User]
export const getCurrentUser = query({
  args: {},
  handler: async (ctx) => {
    // Get authenticated user from session
  },
});
\end{lstlisting}

\textbf{Purpose}: Retrieve current authenticated user information.

\subsection{Mutation: makeAdmin}

\begin{lstlisting}[caption=Promote User to Admin]
export const makeAdmin = mutation({
  args: { userId: v.id("users") },
  handler: async (ctx, args) => {
    // Verify current user is admin
    // Update user role to admin
  },
});
\end{lstlisting}

\textbf{Purpose}: Promote a user to admin role.

\textbf{Authorization}: Requires existing admin role.

\section{Favorites Functions}

\subsection{Query: getUserFavorites}

\begin{lstlisting}[caption=Get User Favorites]
export const getUserFavorites = query({
  args: {},
  handler: async (ctx) => {
    // Get current user
    // Fetch all favorited sites
  },
});
\end{lstlisting}

\textbf{Purpose}: Retrieve all sites favorited by current user.

\subsection{Mutation: addFavorite}

\begin{lstlisting}[caption=Add Site to Favorites]
export const addFavorite = mutation({
  args: { siteId: v.id("heritageSites") },
  handler: async (ctx, args) => {
    // Get current user
    // Create favorite record
  },
});
\end{lstlisting}

\textbf{Purpose}: Add a site to user's favorites.

\subsection{Mutation: removeFavorite}

\begin{lstlisting}[caption=Remove Site from Favorites]
export const removeFavorite = mutation({
  args: { siteId: v.id("heritageSites") },
  handler: async (ctx, args) => {
    // Get current user
    // Delete favorite record
  },
});
\end{lstlisting}

\textbf{Purpose}: Remove a site from user's favorites.

% ============================================================================
% CHAPTER 5: FRONTEND PAGES
% ============================================================================
\chapter{Frontend Pages and Components}

\section{Page Structure}

\subsection{Landing Page (/)}

\textbf{Purpose}: Entry point for the application with hero section and feature showcase.

\textbf{Key Sections}:
\begin{itemize}
    \item Navigation bar with logo and menu
    \item Hero section with rotating background images
    \item Features grid with holographic cards
    \item Call-to-action section
    \item Footer with branding
\end{itemize}

\textbf{Features}:
\begin{itemize}
    \item Particle background animation
    \item Glass morphism effects
    \item Gradient text styling
    \item Responsive navigation
    \item Dynamic authentication buttons
\end{itemize}

\subsection{Explore Page (/explore)}

\textbf{Purpose}: Main discovery interface for browsing heritage sites.

\textbf{Key Features}:
\begin{itemize}
    \item Interactive map with site markers
    \item Grid/list view toggle
    \item Advanced filtering (category, state, UNESCO)
    \item Search functionality
    \item Site cards with primary images
    \item View count and favorite indicators
\end{itemize}

\textbf{Interactions}:
\begin{itemize}
    \item Click site card to view details
    \item Click map marker to navigate to site
    \item Add/remove favorites
    \item Filter and search in real-time
\end{itemize}

\subsection{Site Detail Page (/site/:id)}

\textbf{Purpose}: Comprehensive view of a single heritage site.

\textbf{Content Sections}:
\begin{itemize}
    \item Site header with primary image
    \item Description and historical significance
    \item Media gallery (images, videos, 3D models, panoramas)
    \item Audio guides with language selection
    \item Visitor information (hours, tickets, best time to visit)
    \item Cultural information (folk tales, cuisine, community)
    \item Interactive map showing location
    \item Related sites recommendations
\end{itemize}

\textbf{Interactive Elements}:
\begin{itemize}
    \item 3D model viewer with rotation and zoom
    \item Panorama viewer for 360° views
    \item Audio player with playback controls
    \item Favorite toggle button
    \item Share functionality
\end{itemize}

\subsection{Admin Dashboard (/admin)}

\textbf{Purpose}: Content management interface for administrators.

\textbf{Key Sections}:
\begin{itemize}
    \item Statistics dashboard (total sites, views, audio plays)
    \item Sites management table
    \item Media management interface
    \item Audio guide management
    \item User management
\end{itemize}

\textbf{Admin Capabilities}:
\begin{itemize}
    \item Create, read, update, delete heritage sites
    \item Upload and manage media files
    \item Set primary images for sites
    \item Delete unwanted media
    \item Upload audio guides
    \item Publish/unpublish sites
    \item View analytics and statistics
    \item Manage user roles
\end{itemize}

\subsection{Authentication Page (/auth)}

\textbf{Purpose}: User login and registration interface.

\textbf{Features}:
\begin{itemize}
    \item Email OTP authentication
    \item Anonymous user option
    \item Redirect after authentication
    \item Error handling and validation
\end{itemize}

\subsection{Favorites Page (/favorites)}

\textbf{Purpose}: Display user's saved favorite sites.

\textbf{Features}:
\begin{itemize}
    \item Grid view of favorited sites
    \item Remove from favorites option
    \item Quick navigation to site details
    \item Empty state message
\end{itemize}

\section{Reusable Components}

\subsection{ParticleBackground}

\textbf{Purpose}: Animated particle network background effect.

\textbf{Features}:
\begin{itemize}
    \item Canvas-based particle animation
    \item Particle movement and connection
    \item Responsive sizing
    \item Customizable opacity and color
\end{itemize}

\subsection{HolographicCard}

\textbf{Purpose}: Card component with glass morphism and shimmer effects.

\textbf{Features}:
\begin{itemize}
    \item Glass morphism styling
    \item Holographic border effect
    \item Shimmer animation on hover
    \item Responsive layout
\end{itemize}

\subsection{InteractiveMap}

\textbf{Purpose}: Leaflet-based interactive map for site discovery.

\textbf{Features}:
\begin{itemize}
    \item GeoJSON support for state boundaries
    \item Custom markers for heritage sites
    \item Click handlers for site selection
    \item Zoom and pan controls
    \item State-based filtering
\end{itemize}

\subsection{Model3DViewer}

\textbf{Purpose}: Three.js-based 3D model viewer.

\textbf{Features}:
\begin{itemize}
    \item Model loading and rendering
    \item Rotation and zoom controls
    \item Lighting and material rendering
    \item Responsive sizing
\end{itemize}

\subsection{PanoramaViewer}

\textbf{Purpose}: 360° panoramic view viewer.

\textbf{Features}:
\begin{itemize}
    \item Panorama image rendering
    \item Mouse/touch controls for navigation
    \item Zoom functionality
    \item Fullscreen support
\end{itemize}

% ============================================================================
% CHAPTER 6: ALGORITHMS AND WORKFLOWS
% ============================================================================
\chapter{Algorithms and Workflows}

\section{Site Discovery Algorithm}

\subsection{Filtering and Sorting}

\begin{lstlisting}[caption=Site Discovery Algorithm]
function discoverSites(filters, sortBy) {
  1. Query published sites from database
  2. Apply category filter if specified
  3. Apply state filter if specified
  4. Apply UNESCO filter if specified
  5. Sort by specified criteria (popularity, name, etc.)
  6. Fetch associated media for each site
  7. Return filtered and sorted results
}
\end{lstlisting}

\section{Image Management Workflow}

\subsection{Adding Media}

\begin{enumerate}
    \item Admin uploads media file through dashboard
    \item File is stored in Convex file storage
    \item Media record is created in database
    \item Media is associated with heritage site
    \item UI updates to show new media
\end{enumerate}

\subsection{Setting Primary Image}

\begin{enumerate}
    \item Admin selects image from media gallery
    \item System queries all media for the site
    \item Previous primary image flag is removed
    \item Selected image is marked as primary
    \item Site display updates to show new primary image
\end{enumerate}

\subsection{Deleting Media}

\begin{enumerate}
    \item Admin selects media to delete
    \item System verifies admin authorization
    \item Media record is deleted from database
    \item File is removed from storage
    \item If deleted media was primary, system selects new primary
    \item UI updates to reflect changes
\end{enumerate}

\section{User Authentication Flow}

\begin{enumerate}
    \item User navigates to /auth page
    \item User enters email address
    \item OTP is sent to email
    \item User enters OTP code
    \item System verifies OTP
    \item User session is created
    \item User is redirected to dashboard or explore page
\end{enumerate}

\section{Favorites Management}

\begin{enumerate}
    \item User clicks favorite button on site
    \item System checks if site is already favorited
    \item If not favorited: create favorite record
    \item If favorited: delete favorite record
    \item UI updates to reflect favorite status
    \item Favorites page updates in real-time
\end{enumerate}

% ============================================================================
% CHAPTER 7: STYLING AND THEME
% ============================================================================
\chapter{Styling and Theme System}

\section{Design Philosophy}

VIRASAT employs a futuristic design aesthetic with the following principles:

\begin{itemize}
    \item \textbf{Glass Morphism}: Translucent, frosted glass effects for depth
    \item \textbf{Holographic Elements}: Shimmer and glow effects for visual interest
    \item \textbf{Particle Animations}: Dynamic background elements for engagement
    \item \textbf{Gradient Text}: Multi-color text for branding and emphasis
    \item \textbf{Responsive Design}: Mobile-first approach with desktop enhancements
\end{itemize}

\section{Color Palette}

The application uses OKLCH color format for Tailwind CSS v4:

\begin{itemize}
    \item \textbf{Primary}: Blue-based colors for main UI elements
    \item \textbf{Secondary}: Purple accents for interactive elements
    \item \textbf{Background}: Dark theme with subtle gradients
    \item \textbf{Text}: High contrast for readability
    \item \textbf{Accent}: Cyan/blue for highlights and borders
\end{itemize}

\section{Component Styling}

\subsection{Glass Morphism}

\begin{lstlisting}[caption=Glass Morphism CSS]
.glass-morph {
  background: rgba(255, 255, 255, 0.1);
  backdrop-filter: blur(10px);
  border: 1px solid rgba(255, 255, 255, 0.2);
  border-radius: 8px;
}
\end{lstlisting}

\subsection{Holographic Border}

\begin{lstlisting}[caption=Holographic Border CSS]
.holo-border {
  background: linear-gradient(45deg, #00ff00, #00ffff, #ff00ff);
  padding: 2px;
  border-radius: 8px;
}
\end{lstlisting}

\subsection{Gradient Text}

\begin{lstlisting}[caption=Gradient Text CSS]
.gradient-text {
  background: linear-gradient(135deg, #667eea 0%, #764ba2 100%);
  -webkit-background-clip: text;
  -webkit-text-fill-color: transparent;
  background-clip: text;
}
\end{lstlisting}

% ============================================================================
% CHAPTER 8: PERFORMANCE AND OPTIMIZATION
% ============================================================================
\chapter{Performance and Optimization}

\section{Frontend Optimization}

\subsection{Code Splitting}

\begin{itemize}
    \item Route-based code splitting with React Router
    \item Lazy loading of components
    \item Dynamic imports for heavy libraries
\end{itemize}

\subsection{Image Optimization}

\begin{itemize}
    \item Responsive image sizing
    \item Lazy loading for off-screen images
    \item WebP format support with fallbacks
    \item Image compression and optimization
\end{itemize}

\subsection{Caching Strategy}

\begin{itemize}
    \item Browser caching for static assets
    \item Service worker for offline support
    \item Convex automatic caching for queries
\end{itemize}

\section{Backend Optimization}

\subsection{Database Indexing}

Strategic indexes on frequently queried fields:

\begin{itemize}
    \item \texttt{by\_published}: Fast public site retrieval
    \item \texttt{by\_state}: Geographic queries
    \item \texttt{by\_category}: Category filtering
    \item \texttt{by\_user\_and\_site}: Favorite lookups
\end{itemize}

\subsection{Query Optimization}

\begin{itemize}
    \item Batch queries where possible
    \item Minimize data transfer
    \item Use indexes for filtering
    \item Avoid N+1 query problems
\end{itemize}

\section{Real-Time Performance}

\begin{itemize}
    \item WebSocket connections for live updates
    \item Reactive data subscriptions
    \item Optimistic UI updates
    \item Debounced search and filters
\end{itemize}

% ============================================================================
% CHAPTER 9: SECURITY
% ============================================================================
\chapter{Security and Authorization}

\section{Authentication}

\subsection{Email OTP Authentication}

\begin{itemize}
    \item One-time passwords sent via email
    \item Time-limited OTP validity
    \item Rate limiting on OTP requests
    \item Secure session management
\end{itemize}

\subsection{Session Management}

\begin{itemize}
    \item JWT-based session tokens
    \item Secure cookie storage
    \item Session expiration and refresh
    \item CSRF protection
\end{itemize}

\section{Authorization}

\subsection{Role-Based Access Control}

\begin{itemize}
    \item \textbf{Admin}: Full access to all features and content management
    \item \textbf{User}: Access to exploration and favorites
    \item \textbf{Member}: Limited access to community features
\end{itemize}

\subsection{Function-Level Authorization}

\begin{lstlisting}[caption=Authorization Check Example]
export const create = mutation({
  args: { /* ... */ },
  handler: async (ctx, args) => {
    const user = await getCurrentUser(ctx);
    if (!user || user.role !== "admin") {
      throw new Error("Unauthorized");
    }
    // Proceed with operation
  },
});
\end{lstlisting}

\section{Data Protection}

\begin{itemize}
    \item HTTPS encryption for all communications
    \item Secure file storage with access controls
    \item Input validation and sanitization
    \item SQL injection prevention (NoSQL)
    \item XSS protection through React
\end{itemize}

% ============================================================================
% CHAPTER 10: DEPLOYMENT AND MAINTENANCE
% ============================================================================
\chapter{Deployment and Maintenance}

\section{Development Setup}

\begin{lstlisting}[caption=Development Environment Setup]
# Install dependencies
pnpm install

# Start Convex development server
npx convex dev

# Start Vite development server
pnpm dev

# Run type checker
npx tsc -b --noEmit
\end{lstlisting}

\section{Production Deployment}

\subsection{Frontend Deployment}

\begin{enumerate}
    \item Build optimized production bundle: \texttt{pnpm build}
    \item Deploy to CDN or hosting service
    \item Configure environment variables
    \item Set up SSL/TLS certificates
    \item Configure caching headers
\end{enumerate}

\subsection{Backend Deployment}

\begin{enumerate}
    \item Deploy Convex functions: \texttt{npx convex deploy}
    \item Configure production environment variables
    \item Set up database backups
    \item Configure monitoring and logging
    \item Set up error tracking
\end{enumerate}

\section{Monitoring and Logging}

\begin{itemize}
    \item Application error tracking
    \item Performance monitoring
    \item Database query logging
    \item User activity tracking
    \item Analytics and metrics
\end{itemize}

\section{Maintenance Tasks}

\begin{itemize}
    \item Regular security updates
    \item Dependency updates
    \item Database optimization
    \item Cache invalidation
    \item Backup verification
\end{itemize}

% ============================================================================
% CHAPTER 11: FUTURE ENHANCEMENTS
% ============================================================================
\chapter{Future Enhancements and Roadmap}

\section{Planned Features}

\subsection{Phase 2}

\begin{itemize}
    \item Multi-language support (Hindi, Tamil, Telugu, etc.)
    \item User-generated content and reviews
    \item Social sharing and community features
    \item Advanced search with full-text indexing
    \item Personalized recommendations
\end{itemize}

\subsection{Phase 3}

\begin{itemize}
    \item Mobile native applications (iOS/Android)
    \item Augmented Reality (AR) features
    \item Virtual reality (VR) experiences
    \item Offline mode with service workers
    \item Advanced analytics dashboard
\end{itemize}

\subsection{Phase 4}

\begin{itemize}
    \item AI-powered content recommendations
    \item Chatbot for user assistance
    \item Video streaming optimization
    \item Blockchain-based heritage verification
    \item Integration with tourism boards
\end{itemize}

\section{Technical Improvements}

\begin{itemize}
    \item GraphQL API implementation
    \item Microservices architecture
    \item Advanced caching strategies
    \item Machine learning for personalization
    \item Improved accessibility features
\end{itemize}

% ============================================================================
% APPENDIX
% ============================================================================
\appendix

\chapter{File Structure}

\begin{lstlisting}[caption=Project Directory Structure]
VIRASAT/
├── src/
│   ├── pages/
│   │   ├── Landing.tsx
│   │   ├── Explore.tsx
│   │   ├── SiteDetail.tsx
│   │   ├── AdminDashboard.tsx
│   │   ├── Auth.tsx
│   │   ├── Favorites.tsx
│   │   └── NotFound.tsx
│   ├── components/
│   │   ├── ParticleBackground.tsx
│   │   ├── HolographicCard.tsx
│   │   ├── InteractiveMap.tsx
│   │   ├── Model3DViewer.tsx
│   │   ├── PanoramaViewer.tsx
│   │   ├── FloatingElement.tsx
│   │   ├── AnimatedSection.tsx
│   │   └── ui/ (50+ Shadcn components)
│   ├── convex/
│   │   ├── schema.ts
│   │   ├── heritageSites.ts
│   │   ├── media.ts
│   │   ├── audio.ts
│   │   ├── favorites.ts
│   │   ├── users.ts
│   │   ├── auth.ts
│   │   ├── http.ts
│   │   └── crons.ts
│   ├── hooks/
│   │   ├── use-auth.ts
│   │   └── use-mobile.ts
│   ├── lib/
│   │   ├── utils.ts
│   │   └── downloadSource.ts
│   ├── index.css
│   ├── main.tsx
│   └── instrumentation.tsx
├── public/
│   ├── india.geojson
│   ├── india-states.geojson
│   ├── logo.svg
│   └── logo.png
├── package.json
├── vite.config.ts
├── tsconfig.json
└── README.md
\end{lstlisting}

\chapter{Environment Variables}

\begin{lstlisting}[caption=Required Environment Variables]
# Frontend (.env)
VITE_CONVEX_URL=<your-convex-url>

# Backend (Convex)
CONVEX_DEPLOYMENT=<deployment-id>
UNSPLASH_ACCESS_KEY=<unsplash-api-key>
\end{lstlisting}

\chapter{API Reference}

\section{Heritage Sites API}

\begin{itemize}
    \item \texttt{api.heritageSites.list}: Get published sites
    \item \texttt{api.heritageSites.getById}: Get site details
    \item \texttt{api.heritageSites.search}: Search sites
    \item \texttt{api.heritageSites.create}: Create site (admin)
    \item \texttt{api.heritageSites.update}: Update site (admin)
    \item \texttt{api.heritageSites.remove}: Delete site (admin)
\end{itemize}

\section{Media API}

\begin{itemize}
    \item \texttt{api.media.addMedia}: Add media to site
    \item \texttt{api.media.removeMedia}: Remove media
    \item \texttt{api.media.setPrimaryMedia}: Set primary image
\end{itemize}

\section{Favorites API}

\begin{itemize}
    \item \texttt{api.favorites.getUserFavorites}: Get user favorites
    \item \texttt{api.favorites.addFavorite}: Add to favorites
    \item \texttt{api.favorites.removeFavorite}: Remove from favorites
\end{itemize}

\end{document}
