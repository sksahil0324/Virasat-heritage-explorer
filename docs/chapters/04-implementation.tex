\chapter{Implementation Details}

\section{Frontend Implementation}

\subsection{Landing Page Architecture}

The Landing page serves as the entry point and showcases the platform's futuristic aesthetic.

\subsubsection{Particle Background System}

\begin{lstlisting}[language=JavaScript, caption=Particle Animation Implementation]
// ParticleBackground.tsx
const particles: Particle[] = [];
const particleCount = 80;

// Initialize particles with random positions and velocities
for (let i = 0; i < particleCount; i++) {
  particles.push({
    x: Math.random() * canvas.width,
    y: Math.random() * canvas.height,
    vx: (Math.random() - 0.5) * 0.5,
    vy: (Math.random() - 0.5) * 0.5,
    size: Math.random() * 2 + 1,
    opacity: Math.random() * 0.5 + 0.2,
  });
}

// Animation loop with particle movement and connections
const animate = () => {
  ctx.clearRect(0, 0, canvas.width, canvas.height);
  
  particles.forEach((particle) => {
    // Update position
    particle.x += particle.vx;
    particle.y += particle.vy;
    
    // Wrap around edges
    if (particle.x < 0) particle.x = canvas.width;
    if (particle.x > canvas.width) particle.x = 0;
    if (particle.y < 0) particle.y = canvas.height;
    if (particle.y > canvas.height) particle.y = 0;
    
    // Draw particle
    ctx.beginPath();
    ctx.arc(particle.x, particle.y, particle.size, 0, Math.PI * 2);
    ctx.fillStyle = `rgba(100, 100, 255, ${particle.opacity})`;
    ctx.fill();
  });
  
  // Draw connections between nearby particles
  particles.forEach((p1, i) => {
    particles.slice(i + 1).forEach((p2) => {
      const distance = Math.sqrt(
        (p1.x - p2.x) ** 2 + (p1.y - p2.y) ** 2
      );
      if (distance < 150) {
        ctx.beginPath();
        ctx.moveTo(p1.x, p1.y);
        ctx.lineTo(p2.x, p2.y);
        ctx.strokeStyle = `rgba(100, 100, 255, ${0.15 * (1 - distance / 150)})`;
        ctx.lineWidth = 0.5;
        ctx.stroke();
      }
    });
  });
  
  requestAnimationFrame(animate);
};
\end{lstlisting}

\subsubsection{Rotating Background Images}

\begin{lstlisting}[language=JavaScript, caption=Hero Background Image Rotation]
// Extract images from heritage sites
const monumentImages = sites
  ?.flatMap((site) => 
    site.media
      ?.filter((m) => m.type === "image" && m.storageId)
      .map((m) => ({
        url: m.url,
        name: site.name,
        location: `${site.city}, ${site.state}`,
      }))
  )
  .filter(Boolean) || [];

// Rotate images every 30 seconds
useEffect(() => {
  if (monumentImages.length === 0) return;
  
  const interval = setInterval(() => {
    setCurrentImageIndex((prev) => (prev + 1) % monumentImages.length);
  }, 30000);
  
  return () => clearInterval(interval);
}, [monumentImages.length]);

// Render with AnimatePresence for smooth transitions
<AnimatePresence mode="wait">
  {monumentImages.length > 0 && (
    <motion.div
      key={currentImageIndex}
      initial={{ opacity: 0, scale: 1.1 }}
      animate={{ opacity: 1, scale: 1 }}
      exit={{ opacity: 0, scale: 0.95 }}
      transition={{ duration: 1.5 }}
    >
      <img src={monumentImages[currentImageIndex]?.url} />
    </motion.div>
  )}
</AnimatePresence>
\end{lstlisting}

\subsection{Explore Page Implementation}

\subsubsection{Dynamic Section Rendering}

\begin{lstlisting}[language=JavaScript, caption=URL-Based Section Navigation]
// Get active section from URL params
const [searchParams, setSearchParams] = useSearchParams();
const activeSection = searchParams.get("section") || "explore";

// Handle section changes
const handleSectionChange = (section: string) => {
  setSearchParams({ section });
};

// Conditional rendering based on active section
{activeSection === "explore" && <ExploreSection />}
{activeSection === "map" && <InteractiveMap />}
{activeSection === "360" && <PanoramaSection />}
{activeSection === "gallery" && <GallerySection />}
{activeSection === "stories" && <StoriesSection />}
{activeSection === "community" && <CommunitySection />}
{activeSection === "about" && <AboutSection />}
\end{lstlisting}

\subsubsection{Advanced Filtering System}

\begin{lstlisting}[language=JavaScript, caption=Multi-Criteria Site Filtering]
// Query with filters
const sites = useQuery(api.heritageSites.list, {
  category: category === "all" ? undefined : category,
  state: state === "all" ? undefined : state,
  unescoOnly,
});

// Search functionality
const searchResults = useQuery(
  api.heritageSites.search,
  searchTerm.length > 2 ? { searchTerm } : "skip"
);

// Display appropriate results
const displaySites = searchTerm.length > 2 ? searchResults : sites;
\end{lstlisting}

\subsection{Interactive Map Implementation}

\subsubsection{GeoJSON Integration}

\begin{lstlisting}[language=JavaScript, caption=India States GeoJSON Rendering]
// Load GeoJSON data
useEffect(() => {
  fetch("/india-states.geojson")
    .then((response) => response.json())
    .then((data) => setIndiaGeoJson(data))
    .catch((error) => console.error("Error loading GeoJSON:", error));
}, []);

// Style function for features
const geoJsonStyle = (feature: any) => {
  const isSelected = selectedState === feature.properties.NAME_1;
  return {
    fillColor: isSelected ? "#ffd166" : "#4a6fa5",
    weight: 2,
    opacity: 1,
    color: "white",
    fillOpacity: isSelected ? 0.7 : 0.5,
  };
};

// Event handlers for interactions
const onEachFeature = (feature: any, layer: any) => {
  layer.on({
    mouseover: (e: any) => {
      e.target.setStyle({
        weight: 3,
        fillOpacity: 0.7,
      });
    },
    mouseout: (e: any) => {
      e.target.setStyle(geoJsonStyle(feature));
    },
    click: () => {
      setSelectedState(feature.properties.NAME_1);
    },
  });
};
\end{lstlisting}

\subsubsection{Custom Marker System}

\begin{lstlisting}[language=JavaScript, caption=Dynamic Map Markers]
// Create custom icon based on UNESCO status
const createCustomIcon = (isUNESCO: boolean) => {
  return L.divIcon({
    className: "custom-marker",
    html: `<div style="
      width: 24px;
      height: 24px;
      border-radius: 50%;
      background: ${isUNESCO ? "#ffd166" : "#5bc0be"};
      border: 3px solid white;
      box-shadow: 0 2px 8px rgba(0,0,0,0.3);
      animation: pulse 2s infinite;
    "></div>`,
    iconSize: [24, 24],
    iconAnchor: [12, 12],
  });
};

// Render markers with hover interactions
<Marker
  position={[lat, lng]}
  icon={createCustomIcon(site.isUNESCO)}
  eventHandlers={{
    click: () => setSelectedSite(site),
    mouseover: () => setHoveredSite(site),
    mouseout: () => setHoveredSite(null),
  }}
>
  <Popup>
    <div>
      <h3>{site.name}</h3>
      <p>{site.city}, {site.state}</p>
      <Button onClick={() => navigate(`/site/${site._id}`)}>
        View Details
      </Button>
    </div>
  </Popup>
</Marker>
\end{lstlisting}

\subsubsection{Smooth Map Interactions}

\begin{lstlisting}[language=JavaScript, caption=Optimized Map Configuration]
<MapContainer
  center={[22.9734, 78.6569]}
  zoom={5}
  zoomSnap={0.5}
  zoomDelta={0.5}
  wheelPxPerZoomLevel={120}
  zoomControl={true}
  doubleClickZoom={true}
  touchZoom={true}
  dragging={true}
  zoomAnimation={true}
  fadeAnimation={true}
  markerZoomAnimation={true}
>
  <TileLayer
    url="https://{s}.tile.openstreetmap.org/{z}/{x}/{y}.png"
  />
  <GeoJSON
    data={indiaGeoJson}
    style={geoJsonStyle}
    onEachFeature={onEachFeature}
  />
  {/* Markers */}
</MapContainer>
\end{lstlisting}

\subsection{Site Detail Page}

\subsubsection{3D Model Viewer}

\begin{lstlisting}[language=JavaScript, caption=Three.js 3D Model Rendering]
// Model3DViewer.tsx
import { Canvas } from '@react-three/fiber';
import { OrbitControls, useGLTF } from '@react-three/drei';

function Model({ url }: { url: string }) {
  const { scene } = useGLTF(url);
  
  // Calculate bounding box for proper scaling
  const box = new THREE.Box3().setFromObject(scene);
  const center = box.getCenter(new THREE.Vector3());
  const size = box.getSize(new THREE.Vector3());
  const maxDim = Math.max(size.x, size.y, size.z);
  const scale = 10 / maxDim;
  
  scene.scale.set(scale, scale, scale);
  scene.position.set(-center.x * scale, -center.y * scale, -center.z * scale);
  
  return <primitive object={scene} />;
}

export default function Model3DViewer({ modelUrl }: { modelUrl: string }) {
  return (
    <Canvas camera={{ position: [0, 0, 15], fov: 75 }}>
      <ambientLight intensity={0.5} />
      <directionalLight position={[10, 10, 5]} intensity={1} />
      <Model url={modelUrl} />
      <OrbitControls
        minDistance={3}
        maxDistance={50}
        zoomSpeed={1.5}
        rotateSpeed={1}
        panSpeed={1}
        enableDamping={true}
        dampingFactor={0.05}
      />
    </Canvas>
  );
}
\end{lstlisting}

\subsubsection{360° Panorama Viewer}

\begin{lstlisting}[language=JavaScript, caption=Panoramic Image Viewer]
// PanoramaViewer.tsx
function PanoramaSphere({ imageUrl }: { imageUrl: string }) {
  const texture = useTexture(imageUrl);
  
  return (
    <mesh>
      <sphereGeometry args={[500, 60, 40]} scale={[-1, 1, 1]} />
      <meshBasicMaterial map={texture} side={THREE.BackSide} />
    </mesh>
  );
}

export default function PanoramaViewer({ imageUrl }: { imageUrl: string }) {
  return (
    <Canvas camera={{ position: [0, 0, 0], fov: 75 }}>
      <PanoramaSphere imageUrl={imageUrl} />
      <OrbitControls
        enableZoom={true}
        enablePan={false}
        rotateSpeed={-0.5}
        minDistance={0.1}
        maxDistance={0.1}
      />
    </Canvas>
  );
}
\end{lstlisting}

\section{Backend Implementation}

\subsection{Convex Query Functions}

\subsubsection{Heritage Sites List Query}

\begin{lstlisting}[language=JavaScript, caption=Filtered Sites Query]
export const list = query({
  args: {
    category: v.optional(v.string()),
    state: v.optional(v.string()),
    unescoOnly: v.optional(v.boolean()),
  },
  handler: async (ctx, args) => {
    // Start with published sites index
    let sitesQuery = ctx.db
      .query("heritageSites")
      .withIndex("by_published", (q) => q.eq("isPublished", true));
    
    const sites = await sitesQuery.collect();
    
    // Apply filters
    let filtered = sites;
    if (args.category && args.category !== "all") {
      filtered = filtered.filter((site) => site.category === args.category);
    }
    if (args.state && args.state !== "all") {
      filtered = filtered.filter((site) => site.state === args.state);
    }
    if (args.unescoOnly) {
      filtered = filtered.filter((site) => site.isUNESCO);
    }
    
    // Fetch media for each site
    const sitesWithMedia = await Promise.all(
      filtered.map(async (site) => {
        const media = await ctx.db
          .query("media")
          .withIndex("by_site", (q) => q.eq("siteId", site._id))
          .collect();
        return { ...site, media };
      })
    );
    
    // Sort by popularity
    return sitesWithMedia.sort((a, b) => b.viewCount - a.viewCount);
  },
});
\end{lstlisting}

\subsubsection{Site Detail Query}

\begin{lstlisting}[language=JavaScript, caption=Comprehensive Site Data Retrieval]
export const getById = query({
  args: { id: v.id("heritageSites") },
  handler: async (ctx, args) => {
    const site = await ctx.db.get(args.id);
    if (!site) return null;
    
    // Fetch all related data
    const media = await ctx.db
      .query("media")
      .withIndex("by_site", (q) => q.eq("siteId", args.id))
      .collect();
    
    const audio = await ctx.db
      .query("audioSummaries")
      .withIndex("by_site", (q) => q.eq("siteId", args.id))
      .collect();
    
    return {
      ...site,
      media,
      audio,
    };
  },
});
\end{lstlisting}

\subsection{Mutation Functions}

\subsubsection{Create Heritage Site}

\begin{lstlisting}[language=JavaScript, caption=Site Creation with Authorization]
export const create = mutation({
  args: {
    name: v.string(),
    description: v.string(),
    historicalSignificance: v.string(),
    category: categoryValidator,
    state: v.string(),
    city: v.string(),
    // ... other fields
  },
  handler: async (ctx, args) => {
    // Check authorization
    const user = await getCurrentUser(ctx);
    if (!user || user.role !== "admin") {
      throw new Error("Unauthorized");
    }
    
    // Create site
    const siteId = await ctx.db.insert("heritageSites", {
      ...args,
      viewCount: 0,
      createdBy: user._id,
    });
    
    return siteId;
  },
});
\end{lstlisting}

\subsubsection{Media Upload}

\begin{lstlisting}[language=JavaScript, caption=File Upload with Storage]
export const add = mutation({
  args: {
    siteId: v.id("heritageSites"),
    type: v.union(
      v.literal("image"),
      v.literal("video"),
      v.literal("model3d"),
      v.literal("panorama")
    ),
    storageId: v.optional(v.id("_storage")),
    url: v.string(),
    caption: v.optional(v.string()),
    isPrimary: v.boolean(),
  },
  handler: async (ctx, args) => {
    const user = await getCurrentUser(ctx);
    if (!user || user.role !== "admin") {
      throw new Error("Unauthorized");
    }
    
    const mediaId = await ctx.db.insert("media", args);
    return mediaId;
  },
});

export const generateUploadUrl = mutation({
  args: {},
  handler: async (ctx) => {
    const user = await getCurrentUser(ctx);
    if (!user || user.role !== "admin") {
      throw new Error("Unauthorized");
    }
    
    return await ctx.storage.generateUploadUrl();
  },
});
\end{lstlisting}

\subsection{Favorites System}

\begin{lstlisting}[language=JavaScript, caption=Toggle Favorite Implementation]
export const toggle = mutation({
  args: {
    siteId: v.id("heritageSites"),
  },
  handler: async (ctx, args) => {
    const user = await getCurrentUser(ctx);
    if (!user) {
      throw new Error("Must be logged in to favorite sites");
    }
    
    // Check if already favorited
    const existing = await ctx.db
      .query("favorites")
      .withIndex("by_user_and_site", (q) =>
        q.eq("userId", user._id).eq("siteId", args.siteId)
      )
      .unique();
    
    if (existing) {
      // Remove favorite
      await ctx.db.delete(existing._id);
      return false;
    } else {
      // Add favorite
      await ctx.db.insert("favorites", {
        userId: user._id,
        siteId: args.siteId,
      });
      return true;
    }
  },
});
\end{lstlisting}

\section{UI Component System}

\subsection{Holographic Card Component}

\begin{lstlisting}[language=JavaScript, caption=Animated Card with Glass Morphism]
export default function HolographicCard({ 
  children, 
  className, 
  delay = 0 
}: HolographicCardProps) {
  return (
    <motion.div
      initial={{ opacity: 0, y: 20 }}
      whileInView={{ opacity: 1, y: 0 }}
      viewport={{ once: true }}
      transition={{ duration: 0.6, delay }}
      whileHover={{ scale: 1.02, rotateY: 2 }}
      style={{ transformStyle: "preserve-3d" }}
    >
      <Card className={cn(
        "glass-morph holo-border relative overflow-hidden group", 
        className
      )}>
        <div className="absolute inset-0 shimmer opacity-0 group-hover:opacity-100 transition-opacity duration-500" />
        <div className="relative z-10">{children}</div>
      </Card>
    </motion.div>
  );
}
\end{lstlisting}

\subsection{Floating Element Animation}

\begin{lstlisting}[language=JavaScript, caption=Continuous Float Animation]
export default function FloatingElement({ 
  children, 
  delay = 0, 
  duration = 6 
}: FloatingElementProps) {
  return (
    <motion.div
      initial={{ y: 0 }}
      animate={{
        y: [-10, 10, -10],
        rotateZ: [-2, 2, -2],
      }}
      transition={{
        duration,
        delay,
        repeat: Infinity,
        ease: "easeInOut",
      }}
      style={{ transformStyle: "preserve-3d" }}
    >
      {children}
    </motion.div>
  );
}
\end{lstlisting}

\section{Authentication Implementation}

\subsection{Custom Auth Hook}

\begin{lstlisting}[language=JavaScript, caption=useAuth Hook]
export function useAuth() {
  const { isLoading: isAuthLoading, isAuthenticated } = useConvexAuth();
  const user = useQuery(api.users.currentUser);
  const authActions = useAuthActions();
  const [isLoading, setIsLoading] = useState(true);
  
  useEffect(() => {
    if (!isAuthLoading && user !== undefined) {
      setIsLoading(false);
    }
  }, [isAuthLoading, user]);
  
  return {
    isLoading,
    isAuthenticated,
    user,
    signIn: authActions.signIn,
    signOut: authActions.signOut,
  };
}
\end{lstlisting}

\section{Performance Optimizations}

\subsection{Image Loading with Error Handling}

\begin{lstlisting}[language=JavaScript, caption=Robust Image Display]
<img
  src={primaryImage.url}
  alt={site.name}
  onError={(e) => {
    console.error(`Failed to load image for ${site.name}`);
    e.currentTarget.style.display = 'none';
    const parent = e.currentTarget.parentElement;
    if (parent) {
      parent.innerHTML = `
        <div class="aspect-video bg-muted flex items-center justify-center">
          <svg><!-- Fallback SVG icon --></svg>
        </div>
      `;
    }
  }}
/>
\end{lstlisting}

\subsection{Debounced Search}

\begin{lstlisting}[language=JavaScript, caption=Search Optimization]
const searchResults = useQuery(
  api.heritageSites.search,
  searchTerm.length > 2 ? { searchTerm } : "skip"
);
\end{lstlisting}

\section{Responsive Design Implementation}

\subsection{Tailwind Responsive Classes}

\begin{lstlisting}[language=HTML, caption=Mobile-First Responsive Grid]
<div className="grid grid-cols-1 md:grid-cols-2 lg:grid-cols-3 gap-6">
  {sites.map((site) => (
    <Card key={site._id}>
      {/* Card content */}
    </Card>
  ))}
</div>
\end{lstlisting}

\subsection{Mobile Navigation}

\begin{lstlisting}[language=JavaScript, caption=Responsive Navigation Menu]
<div className="hidden lg:flex items-center gap-1">
  {menuItems.map((item) => (
    <Button variant="ghost" onClick={() => navigate(item.path)}>
      {item.label}
    </Button>
  ))}
</div>
\end{lstlisting}

\section{Error Handling and Logging}

\subsection{Comprehensive Error Boundaries}

\begin{lstlisting}[language=JavaScript, caption=Error Logging Strategy]
try {
  const result = await uploadFile(file, uploadUrl);
  toast.success(`${file.name} uploaded successfully`);
} catch (error) {
  console.error("Upload failed:", error);
  toast.error(`Failed to upload ${file.name}`);
}
\end{lstlisting}

\section{Analytics and Tracking}

\subsection{View Count Tracking}

\begin{lstlisting}[language=JavaScript, caption=Automatic View Tracking]
useEffect(() => {
  if (site) {
    incrementViewCount({ id: site._id });
  }
}, [site?._id]);
\end{lstlisting}

\subsection{Admin Statistics}

\begin{lstlisting}[language=JavaScript, caption=Platform Analytics]
export const getStats = query({
  args: {},
  handler: async (ctx) => {
    const user = await getCurrentUser(ctx);
    if (!user || user.role !== "admin") {
      throw new Error("Unauthorized");
    }
    
    const sites = await ctx.db.query("heritageSites").collect();
    const totalViews = sites.reduce((sum, site) => sum + site.viewCount, 0);
    
    const audio = await ctx.db.query("audioSummaries").collect();
    const totalPlays = audio.reduce((sum, a) => sum + a.playCount, 0);
    
    return {
      totalSites: sites.length,
      publishedSites: sites.filter((s) => s.isPublished).length,
      totalViews,
      totalAudioPlays: totalPlays,
      unescoSites: sites.filter((s) => s.isUNESCO).length,
    };
  },
});
\end{lstlisting}
